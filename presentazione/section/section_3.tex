% --- SEZIONE 3: ANALISI DATI ESTESA ---
\section{Analisi dei Dati}

% ==============================================================================
% SUBTASK A
% ==============================================================================

% --- SLIDE A1: TRAIN ---
\begin{frame}{Task A [Train]: Learning Source Bias}
    \vspace{-0.3cm}
    Analisi del Training Set (500k sample). Notiamo un forte bias verso linguaggi popolari e generatori specifici.
    \vspace{0.1cm}

    \begin{columns}[t, onlytextwidth]
        % Colonna 1: Linguaggi
        \column{0.32\textwidth}
        \begin{tcolorbox}[enhanced, title=\faCode\ Languages, colframe=Obsidian, colback=white, coltitle=white, boxed title style={colback=VividViolet}, fonttitle=\bfseries\tiny, top=0pt, bottom=0pt]
            \centering
            \includegraphics[width=\linewidth, height=2.5cm, keepaspectratio]{../img/img_TaskA/Train_label_language.png}
            \tiny Dominio: Python/Java.
        \end{tcolorbox}

        % Colonna 2: Lunghezza
        \column{0.32\textwidth}
        \begin{tcolorbox}[enhanced, title=\faRuler\ Length, colframe=Obsidian, colback=white, coltitle=white, boxed title style={colback=CyanGlow!80!blue}, fonttitle=\bfseries\tiny, top=0pt, bottom=0pt]
            \centering
            \includegraphics[width=\linewidth, height=2.5cm, keepaspectratio]{../img/img_TaskA/Train_length_label.png}
            \tiny AI è più verbosa (Code Llama).
        \end{tcolorbox}

        % Colonna 3: Generator
        \column{0.32\textwidth}
        \begin{tcolorbox}[enhanced, title=\faRobot\ Generators, colframe=Obsidian, colback=white, coltitle=white, boxed title style={colback=DeepViolet}, fonttitle=\bfseries\tiny, top=0pt, bottom=0pt]
            \centering
            \includegraphics[width=\linewidth, height=2.5cm, keepaspectratio]{../img/img_TaskA/Train_top_generators.png}
            \tiny GPT-4 e Llama dominanti.
        \end{tcolorbox}
    \end{columns}
    
    \vspace{0.1cm}
    \begin{tcolorbox}[colback=gray!10, frame hidden, left=2pt, right=2pt, top=1pt, bottom=1pt]
        \centering \scriptsize \textbf{Insight:} Il modello rischia di sovra-adattarsi allo stile di GPT-4 se non regolarizzato.
    \end{tcolorbox}
\end{frame}

% --- SLIDE A2: VALIDATION ---
\begin{frame}{Task A [Validation]: Consistency Check}
    \vspace{-0.3cm}
    Verifica che il Validation Set rispecchi la distribuzione del Train per evitare il "Data Leakage" o shift distribuzionali.
    \vspace{0.1cm}

    \begin{columns}[t, onlytextwidth]
        \column{0.49\textwidth}
        \begin{tcolorbox}[enhanced, title=\faSync\ Label Consistency, colframe=Obsidian, colback=white, coltitle=white, boxed title style={colback=Charcoal}, fonttitle=\bfseries\scriptsize, top=0pt, bottom=0pt]
            \centering
            \includegraphics[width=\linewidth, height=3.2cm, keepaspectratio]{../img/img_TaskA/Validation_label_language.png}
        \end{tcolorbox}

        \column{0.49\textwidth}
        \begin{tcolorbox}[enhanced, title=\faChartBar\ Validation Generators, colframe=Obsidian, colback=white, coltitle=white, boxed title style={colback=VividViolet}, fonttitle=\bfseries\scriptsize, top=0pt, bottom=0pt]
            \centering
            \includegraphics[width=\linewidth, height=3.2cm, keepaspectratio]{../img/img_TaskA/Validation_top_generators.png}
        \end{tcolorbox}
    \end{columns}
\end{frame}

% --- SLIDE A3: TEST ---
\begin{frame}{Task A [Test]: The "Unseen" Challenge}
    \vspace{-0.3cm}
    Il Test Set introduce la vera difficoltà: linguaggi e generatori \textbf{MAI VISTI} durante il training.
    \vspace{0.1cm}

    \begin{columns}[t, onlytextwidth]
        % Colonna SX: Grafico Test
        \column{0.55\textwidth}
        \begin{tcolorbox}[enhanced, title=\faExclamationTriangle\ OOD Languages, colframe=Obsidian, colback=white, coltitle=white, boxed title style={colback=CyanGlow!80!blue}, fonttitle=\bfseries\scriptsize, top=0pt, bottom=0pt]
            \centering
            \includegraphics[width=\linewidth, keepaspectratio]{../img/img_TaskA/Test_label_language.png}
        \end{tcolorbox}

        % Colonna DX: Spiegazione
        \column{0.43\textwidth}
        \begin{tcolorbox}[colback=gray!5, colframe=Obsidian, title=\textbf{Impact Analysis}, fonttitle=\bfseries\scriptsize]
            \scriptsize
            Il grafico mostra la comparsa di:
            \begin{itemize}
                \item \textbf{Go, PHP, C\#}: Assenti nel train.
                \item \textbf{Obiettivo:} Valutare se il modello ha imparato la "struttura dell'AI" o solo la "sintassi di Python".
            \end{itemize}
            \vspace{0.1cm}
            \textit{Ecco perché usiamo feature agnostiche come la \textbf{Perplexity}.}
        \end{tcolorbox}
    \end{columns}
\end{frame}


% ==============================================================================
% SUBTASK B
% ==============================================================================

% --- SLIDE B1: TRAIN ---
\begin{frame}{Task B [Train]: Fingerprinting Analysis}
    \vspace{-0.3cm}
    Analisi delle caratteristiche uniche dei modelli AI per l'attribuzione di paternità.
    \vspace{0.1cm}

    \begin{columns}[t, onlytextwidth]
        \column{0.49\textwidth}
        \begin{tcolorbox}[enhanced, title=\faBalanceScale\ Extreme Imbalance, colframe=Obsidian, colback=white, coltitle=white, boxed title style={colback=Obsidian}, fonttitle=\bfseries\scriptsize, top=0pt, bottom=0pt]
            \centering
            \includegraphics[width=\linewidth, height=3.0cm, keepaspectratio]{../img/img_TaskB/Train_class_dist.png}
            \tiny Human (90\%) oscura le 10 classi AI.
        \end{tcolorbox}

        \column{0.49\textwidth}
        \begin{tcolorbox}[enhanced, title=\faAlignLeft\ Verbosity (Tokens), colframe=Obsidian, colback=white, coltitle=white, boxed title style={colback=VividViolet}, fonttitle=\bfseries\scriptsize, top=0pt, bottom=0pt]
            \centering
            \includegraphics[width=\linewidth, height=3.0cm, keepaspectratio]{../img/img_TaskB/Train_token_boxplot.png}
            \tiny GPT-4 (verbose) vs Llama (conciso).
        \end{tcolorbox}
    \end{columns}
    
    \vspace{0.1cm}
    \centering \scriptsize \textbf{Strategy:} Utilizzo di \textbf{Focal Loss} per le classi rare e feature di lunghezza per distinguere i modelli.
\end{frame}

% --- SLIDE B2: VALIDATION ---
\begin{frame}{Task B [Validation]: Correlation Map}
    \vspace{-0.3cm}
    Analisi delle correlazioni condizionate: "Quale modello scrive in quale linguaggio?"
    \vspace{0.1cm}

    \begin{center}
        \begin{tcolorbox}[enhanced, title=\faBraille\ Generator-Language Heatmap, colframe=Obsidian, colback=white, coltitle=white, boxed title style={colback=DeepViolet}, fonttitle=\bfseries\small, width=0.8\textwidth, top=0pt, bottom=0pt]
            \centering
            \includegraphics[width=\linewidth, height=4.5cm, keepaspectratio]{../img/img_TaskB/Validation_heatmap_norm.png}
        \end{tcolorbox}
    \end{center}
    \vspace{-0.2cm}
    \scriptsize \centering Alcuni modelli sono specializzati (es. StarCoder su Java), creando pattern riconoscibili.
\end{frame}

% --- SLIDE B3: TEST ---
\begin{frame}{Task B [Test]: Robustness Verification}
    \vspace{-0.3cm}
    Verifica finale sulla distribuzione dei token nel set di Test per confermare l'assenza di drift anomali.
    \vspace{0.1cm}

    \begin{columns}[t, onlytextwidth]
        \column{0.49\textwidth}
        \begin{tcolorbox}[enhanced, title=\faChartPie\ Class Distribution (Test), colframe=Obsidian, colback=white, coltitle=white, boxed title style={colback=Charcoal}, fonttitle=\bfseries\scriptsize, top=0pt, bottom=0pt]
            \centering
            \includegraphics[width=\linewidth, height=3.2cm, keepaspectratio]{../img/img_TaskB/Test_class_dist.png}
        \end{tcolorbox}

        \column{0.49\textwidth}
        \begin{tcolorbox}[enhanced, title=\faWaveSquare\ Token Distribution, colframe=Obsidian, colback=white, coltitle=white, boxed title style={colback=CyanGlow!80!blue}, fonttitle=\bfseries\scriptsize, top=0pt, bottom=0pt]
            \centering
            \includegraphics[width=\linewidth, height=3.2cm, keepaspectratio]{../img/img_TaskB/Test_token_dist_general.png}
        \end{tcolorbox}
    \end{columns}
\end{frame}


% ==============================================================================
% SUBTASK C
% ==============================================================================

% --- SLIDE C1: TRAIN ---
\begin{frame}{Task C [Train]: The 4-Class Problem}
    \vspace{-0.3cm}
    Analisi delle classi complesse: \textbf{Hybrid} (Uomo+AI) e \textbf{Adversarial} (Offuscato).
    \vspace{0.1cm}

    \begin{columns}[t, onlytextwidth]
        \column{0.49\textwidth}
        \begin{tcolorbox}[enhanced, title=\faBullseye\ Target Distribution, colframe=Obsidian, colback=white, coltitle=white, boxed title style={colback=CyanGlow!80!blue}, fonttitle=\bfseries\scriptsize, top=0pt, bottom=0pt]
            \centering
            \includegraphics[width=\linewidth, height=3.0cm, keepaspectratio]{../img/img_TaskC/Train_target_dist.png}
            \tiny Hybrid e Adversarial sono minoritarie.
        \end{tcolorbox}

        \column{0.49\textwidth}
        \begin{tcolorbox}[enhanced, title=\faLanguage\ Language Variety, colframe=Obsidian, colback=white, coltitle=white, boxed title style={colback=VividViolet}, fonttitle=\bfseries\scriptsize, top=0pt, bottom=0pt]
            \centering
            \includegraphics[width=\linewidth, height=3.0cm, keepaspectratio]{../img/img_TaskC/Train_languages.png}
            \tiny Elevata frammentazione linguistica.
        \end{tcolorbox}
    \end{columns}
    \vspace{0.1cm}
    \centering \scriptsize \textbf{Solution:} \textit{Structural Noise Injection} per aumentare i campioni rari.
\end{frame}

% --- SLIDE C2: VALIDATION ---
\begin{frame}{Task C [Validation]: Length \& Features}
    \vspace{-0.3cm}
    Controllo della distribuzione delle lunghezze per settare correttamente la \textit{max\_length} del Transformer (GraphCodeBERT).
    \vspace{0.1cm}

    \begin{columns}[t, onlytextwidth]
        \column{0.49\textwidth}
        \begin{tcolorbox}[enhanced, title=\faRulerHorizontal\ Length Distribution (Val), colframe=Obsidian, colback=white, coltitle=white, boxed title style={colback=Obsidian}, fonttitle=\bfseries\scriptsize, top=0pt, bottom=0pt]
            \centering
            \includegraphics[width=\linewidth, height=3.2cm, keepaspectratio]{../img/img_TaskC/Validation_length_dist.png}
        \end{tcolorbox}

        \column{0.49\textwidth}
        \begin{tcolorbox}[enhanced, title=\faCheckDouble\ Target Check (Val), colframe=Obsidian, colback=white, coltitle=white, boxed title style={colback=DeepViolet}, fonttitle=\bfseries\scriptsize, top=0pt, bottom=0pt]
            \centering
            \includegraphics[width=\linewidth, height=3.2cm, keepaspectratio]{../img/img_TaskC/Validation_target_dist.png}
        \end{tcolorbox}
    \end{columns}
\end{frame}

% --- SLIDE C3: TEST ---
\begin{frame}{Task C [Test]: Preparing for Submission}
    \vspace{-0.3cm}
    Analisi finale sul Sample Test Set per garantire che la pipeline di inferenza gestisca correttamente i dati non visti.
    \vspace{0.1cm}

    \begin{columns}[t, onlytextwidth]
        \column{0.32\textwidth}
        \begin{tcolorbox}[enhanced, title=\faCode\ Test Langs, colframe=Obsidian, colback=white, coltitle=white, boxed title style={colback=Charcoal}, fonttitle=\bfseries\tiny, top=0pt, bottom=0pt]
            \centering
            \includegraphics[width=\linewidth, height=2.5cm, keepaspectratio]{../img/img_TaskC/Test_Sample_languages.png}
        \end{tcolorbox}

        \column{0.32\textwidth}
        \begin{tcolorbox}[enhanced, title=\faRuler\ Test Length, colframe=Obsidian, colback=white, coltitle=white, boxed title style={colback=VividViolet}, fonttitle=\bfseries\tiny, top=0pt, bottom=0pt]
            \centering
            \includegraphics[width=\linewidth, height=2.5cm, keepaspectratio]{../img/img_TaskC/Test_Sample_length_dist.png}
        \end{tcolorbox}
        
        \column{0.32\textwidth}
        \begin{tcolorbox}[enhanced, title=\faBullseye\ Test Target, colframe=Obsidian, colback=white, coltitle=white, boxed title style={colback=CyanGlow!80!blue}, fonttitle=\bfseries\tiny, top=0pt, bottom=0pt]
            \centering
            \includegraphics[width=\linewidth, height=2.5cm, keepaspectratio]{../img/img_TaskC/Test_Sample_target_dist.png}
        \end{tcolorbox}
    \end{columns}
\end{frame}