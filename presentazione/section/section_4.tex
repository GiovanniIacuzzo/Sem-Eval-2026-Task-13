% --- SEZIONE 4: RISULTATI DEL TASK ---
\section{Risultati del Task}

% ---------------------------------------------------------
% 4.1 SUBTASK A (Binary Classification)
% ---------------------------------------------------------
\subsection*{Subtask A: Performance Analysis}

\begin{frame}{Subtask A: Training Dynamics \& F1 Score}
    \small
    \textbf{Analisi:} Sui dati di validazione (linguaggi noti), il modello raggiunge performance quasi perfette.
    \vspace{0.1cm}

    \begin{columns}[c]
        \begin{column}{0.5\textwidth}
            \begin{tcolorbox}[colback=white, colframe=gray!20, boxrule=0.5pt, sharp corners]
                \centering
                \includegraphics[width=\textwidth, height=3.0cm, keepaspectratio]{img/results_subtask_a/Val_f1_macro VS step.jpeg}
            \end{tcolorbox}
            \centering \tiny \textit{Validation F1-Macro (Peak: \textbf{0.993})}
        \end{column}
        \pause
        \begin{column}{0.48\textwidth}
            \begin{tcolorbox}[enhanced, title=Convergenza Ottimale, colframe=VividViolet, colback=white, fonttitle=\bfseries\footnotesize, drop shadow]
                \footnotesize
                \begin{itemize}
                    \item \textbf{High Fidelity:} Apprende rapidamente i pattern sintattici di Python/Java.
                    \item \textbf{Stability:} Validation loss coerente.
                \end{itemize}
                
                \tcblower
                
                \textbf{Risultato:}
                \\ Il task Human vs AI su linguaggi visti è \textbf{risolto} (F1 $> 0.99$).
            \end{tcolorbox}
        \end{column}
    \end{columns}
\end{frame}

\begin{frame}{Subtask A: Precisione Estrema (Validation)}
    \small
    La matrice conferma che il modello ha "decodificato" perfettamente lo stile dei linguaggi di training.
    \vspace{0.1cm}

    \begin{columns}[c]
        \begin{column}{0.45\textwidth}
            \centering
            \begin{tcolorbox}[colback=white, colframe=gray!20, boxrule=1pt, sharp corners]
                \centering
                \includegraphics[width=\textwidth, height=3.6cm, keepaspectratio]{img/results_subtask_a/confusion_matrix.png}
            \end{tcolorbox}
        \end{column}
        \pause
        \begin{column}{0.53\textwidth}
            \begin{tcolorbox}[enhanced, title=\faCheckDouble\ Near-Zero Errors, colframe=Obsidian, colback=white, fonttitle=\bfseries\footnotesize, drop shadow]
                \footnotesize
                Su oltre 30.000 campioni:
                \begin{itemize}
                    \item \textbf{False Positives:} Solo $\approx 130$.
                    \item \textbf{False Negatives:} Solo $\approx 80$.
                \end{itemize}
                
                \tcblower
                
                \faInfoCircle\ \textbf{Interpretazione:}
                \\ Il modello identifica "impronte digitali" forti (es. indentazione) che separano Umano da AI.
            \end{tcolorbox}
        \end{column}
    \end{columns}
\end{frame}

\begin{frame}{Subtask A: The Generalization Gap (Test Set)}
    \small
    \textbf{Il Paradosso:} Training perfetto vs crollo delle performance sui dati di Test ciechi (Kaggle).
    \vspace{0.2cm}
    \begin{columns}[t]
        \begin{column}{0.48\textwidth}
            \begin{tcolorbox}[enhanced, colback=VividViolet!10, colframe=VividViolet, title=Validation (Python/Java), fonttitle=\bfseries\footnotesize, halign=center]
                \Large \textbf{0.993}
                \\ \footnotesize F1-Macro
                \\ \scriptsize \faCheck\ \textit{Seen Syntax}
            \end{tcolorbox}
        \end{column}
        \pause
        \begin{column}{0.48\textwidth}
            \begin{tcolorbox}[enhanced, colback=red!10, colframe=red!60!black, title=Test (Go/PHP/...), fonttitle=\bfseries\footnotesize, halign=center]
                \Large \textbf{$\approx$ 0.45}
                \\ \footnotesize F1-Macro
                \\ \scriptsize \faExclamationTriangle\ \textit{Unseen Syntax}
            \end{tcolorbox}
        \end{column}
    \end{columns}
    \vspace{0.2cm}
    
    \begin{tcolorbox}[colback=gray!5, colframe=Obsidian, title=\faSearch\ Root Cause: Overfitting Sintattico, fonttitle=\bfseries\footnotesize]
        \footnotesize
        Il modello non ha appreso la semantica universale, ma ha memorizzato le keyword.
        \begin{itemize}
            \item \textbf{Learned:} \textit{"\texttt{def} + indentazione = Python AI"}.
            \item \textbf{Failed:} Nuovi costrutti (es. \texttt{func} in Go, \texttt{\$} in PHP) causano il fallimento del transfer learning.
        \end{itemize}
    \end{tcolorbox}
\end{frame}

% ---------------------------------------------------------
% 4.2 SUBTASK B (Multi-Class & Cascade)
% ---------------------------------------------------------
\subsection*{Subtask B: Performance Analysis}

\begin{frame}{Subtask B (Stage 1): Il "Gatekeeper" Binario}
    \small
    Il filtro binario raggiunge rapidamente il picco di performance. L'Early Stopping è cruciale.
    \vspace{0.1cm}

    \begin{columns}[c]
        \begin{column}{0.5\textwidth}
            \begin{tcolorbox}[colback=white, colframe=gray!20, boxrule=0.5pt, sharp corners]
                \centering
                \includegraphics[width=\textwidth, height=3.0cm, keepaspectratio]{img/results_subtask_b/binary/Val_f1_macro VS step.jpeg}
            \end{tcolorbox}
            \centering \tiny \textit{Validation F1 (Peak at Epoch 0)}
        \end{column}

        \begin{column}{0.48\textwidth}
            \begin{tcolorbox}[enhanced, title=\faShield* Robustezza del Filtro, colframe=Obsidian, colback=white, fonttitle=\bfseries\footnotesize, drop shadow]
                \footnotesize
                \begin{itemize}
                    \item \textbf{Convergenza Lampo:} F1 > 0.93 alla prima epoca.
                    \item \textbf{Strategia:} Usiamo il checkpoint migliore per evitare il degrado successivo.
                \end{itemize}
                
                \tcblower
                
                \textbf{Impatto Pipeline:}
                \\ Filtra aggressivamente i codici umani, proteggendo il classificatore downstream.
            \end{tcolorbox}
        \end{column}
    \end{columns}
\end{frame}

\begin{frame}{Subtask B (Stage 2): Family Classification}
    \small
    La **SupCon Loss** ha creato cluster ben definiti per le 11 famiglie, come visibile dalla diagonale.
    \vspace{0.1cm}

    \begin{columns}[c]
        \begin{column}{0.45\textwidth}
            \centering
            \begin{tcolorbox}[colback=white, colframe=gray!20, boxrule=1pt, sharp corners]
                \centering
                \includegraphics[width=\textwidth, height=3.6cm, keepaspectratio]{img/results_subtask_b/families/families_cm.png}
            \end{tcolorbox}
        \end{column}

        \begin{column}{0.53\textwidth}
            \begin{tcolorbox}[enhanced, title=\faFingerprint\ Fingerprinting Riuscito, colframe=VividViolet, colback=white, fonttitle=\bfseries\footnotesize, drop shadow]
                \footnotesize
                \begin{itemize}
                    \item \textbf{Diagonale Netta:} GPT, Llama e Granite sono identificati con altissima precisione.
                    \item \textbf{Errori Coerenti:} Le confusioni (es. Mistral vs Llama) avvengono tra architetture "cugine".
                \end{itemize}
                
                \tcblower
                \faChartLine\ \textbf{Metric Learning:}
                \\ Il modello non memorizza solo le keyword, ma apprende lo stile strutturale.
            \end{tcolorbox}
        \end{column}
    \end{columns}
\end{frame}

\begin{frame}{Subtask B: Risultati Competitivi (Test Set)}
    \small
    Qui la generalizzazione è eccellente. Il gap tra Validation e Test è minimo.

    \begin{columns}[t]
        \begin{column}{0.48\textwidth}
            \begin{tcolorbox}[enhanced, colback=CyanGlow!10, colframe=CyanGlow!80!black, title=Kaggle Leaderboard, fonttitle=\bfseries\footnotesize, halign=center]
                \Large \textbf{7\textsuperscript{th}} \footnotesize Place
                \vspace{0.1cm}
                \\ \scriptsize \faTrophy\ \textit{Top Tier Performance}
                \\ \tiny (Gap dal 5\textsuperscript{o}: $\approx 0.001$)
            \end{tcolorbox}
        \end{column}
        
        \begin{column}{0.48\textwidth}
            \begin{tcolorbox}[enhanced, colback=white, colframe=Obsidian, title=Metriche Finali, fonttitle=\bfseries\footnotesize]
                \footnotesize
                \begin{itemize}
                    \item \textbf{F1-Macro:} Allineato con lo State-of-the-Art.
                    \item \textbf{Robustezza:} Il modello funziona anche su dati "in the wild".
                \end{itemize}
            \end{tcolorbox}
        \end{column}
    \end{columns}

    \vspace{0.2cm}
    \begin{tcolorbox}[colback=gray!5, colframe=VividViolet, title=\faLightbulb\ Winning Factors, fonttitle=\bfseries\footnotesize]
        \footnotesize
        \textbf{Architettura a Cascata + SupCon:}
        \begin{itemize}
            \item \textbf{Divide et Impera:} Separare Binary e Multi-class riduce il rumore.
            \item \textbf{Deep Style:} Il Metric Learning cattura la struttura profonda, superando l'overfitting superficiale del Task A.
        \end{itemize}
    \end{tcolorbox}
\end{frame}

% ---------------------------------------------------------
% 4.3 SUBTASK C (Hybrid & Adversarial)
% ---------------------------------------------------------
\subsection*{Subtask C: Performance Analysis}

% SLIDE 1: BINARY MODEL (Problemi Nascosti)
\begin{frame}{Subtask C (Binary): Human vs Machine}
    \small
    Il modello binario mostra ottime metriche di validazione, ma l'approccio basato su feature statiche ha dei limiti intrinseci.
    \vspace{0.1cm}

    \begin{columns}[c]
        \begin{column}{0.5\textwidth}
            \begin{tcolorbox}[colback=white, colframe=gray!20, boxrule=0.5pt, sharp corners]
                \centering
                % Uso le virgolette per gestire gli spazi nel nome file
                \includegraphics[width=\textwidth, height=3.0cm, keepaspectratio]{"img/results_subtask_c/binary/Val_f1_macro VS step.jpeg"}
            \end{tcolorbox}
            \centering \tiny \textit{Binary F1 (Validation)}
        \end{column}
        \pause
        \begin{column}{0.48\textwidth}
            \begin{tcolorbox}[enhanced, title=\faExclamationTriangle\ Fragilità Strutturale, colframe=Obsidian, colback=white, fonttitle=\bfseries\footnotesize, drop shadow]
                \footnotesize
                \begin{itemize}
                    \item \textbf{High Validation:} F1 alto sui dati noti.
                    \item \textbf{Il Problema:} Il modello si affida troppo a picchi di entropia standard per rilevare le macchine.
                \end{itemize}
                
                \tcblower
                
                \textbf{Limite:}
                \\ Se l'attaccante cambia metodo di offuscamento (es. riduce l'entropia artificialmente), il classificatore binario diventa cieco.
            \end{tcolorbox}
        \end{column}
    \end{columns}
\end{frame}

% SLIDE 2: ATTRIBUTION MODEL (Punti di Forza)
\begin{frame}{Subtask C (Attribution): AI vs Hybrid vs Adv}
    \small
    Qui l'architettura mostra i suoi muscoli: la separazione fine-grained è molto stabile durante il training.
    \vspace{0.1cm}

    \begin{columns}[c]
        \begin{column}{0.5\textwidth}
            \begin{tcolorbox}[colback=white, colframe=gray!20, boxrule=0.5pt, sharp corners]
                \centering
                % Uso le virgolette per gestire gli spazi nel nome file
                \includegraphics[width=\textwidth, height=3.0cm, keepaspectratio]{"img/results_subtask_c/machine_attribution/Val_f1_macro VS step.jpeg"}
            \end{tcolorbox}
            \centering \tiny \textit{Attribution F1-Macro (Validation)}
        \end{column}
        \pause
        \begin{column}{0.48\textwidth}
            \begin{tcolorbox}[enhanced, title=\faCheckCircle\ Training Robusto, colframe=VividViolet, colback=white, fonttitle=\bfseries\footnotesize, drop shadow]
                \footnotesize
                \begin{itemize}
                    \item \textbf{Convergenza:} Nessuna oscillazione, segno che la SupCon Loss sta lavorando bene sui cluster.
                    \item \textbf{Hybrid Detection:} Il modello riesce a distinguere efficacemente il codice ibrido da quello puramente generato nei dati di training.
                \end{itemize}
                
                \tcblower
                \textbf{Punto di Forza:}
                \\ L'architettura è capace di apprendere sfumature complesse quando i pattern sono rappresentati nel dataset.
            \end{tcolorbox}
        \end{column}
    \end{columns}
\end{frame}

% SLIDE 3: IL CONFRONTO TEST (Top 5 vs Crollo)
\begin{frame}{Subtask C: The "Scale Shock" (Test Results)}
    \small
    Il confronto tra il test preliminare e quello finale evidenzia un drastico \textbf{Distribution Shift}.
    \vspace{0.1cm}

    \begin{columns}[t]
        \begin{column}{0.48\textwidth}
            % Box Verde: Successo Iniziale
            \begin{tcolorbox}[enhanced, colback=CyanGlow!10, colframe=CyanGlow!80!black, title={Sample Test (Top 5)}, fonttitle=\bfseries\scriptsize, halign=center]
                \Large \textbf{$\approx$ 0.85}
                \\ \scriptsize F1-Macro
                \\ \tiny \faTrophy\ \textit{Leaderboard Preliminare}
            \end{tcolorbox}
        \end{column}
        \pause
        \begin{column}{0.48\textwidth}
            % Box Rosso: Crollo Finale
            \begin{tcolorbox}[enhanced, colback=red!10, colframe=red!60!black, title={Full Test (500k)}, fonttitle=\bfseries\scriptsize, halign=center]
                \Large \textbf{$\approx$ 0.45}
                \\ \scriptsize F1-Macro
                \\ \tiny \faBomb\ \textit{New Adversarial Methods}
            \end{tcolorbox}
        \end{column}
    \end{columns}

    \vspace{0.1cm}
    
    \begin{tcolorbox}[colback=gray!5, colframe=Obsidian, title=\faSearch\ Diagnosi: Cosa è successo?, fonttitle=\bfseries\footnotesize]
        \footnotesize
        Nonostante la forza del modello Attribution (Slide 2), il Full Test ha introdotto scenari imprevisti:
        \begin{itemize}
            \item \textbf{Scale Shock:} Su 500k esempi, il rumore statistico aumenta esponenzialmente.
            \item \textbf{Unseen Attacks:} Probabile presenza di attacchi avversari non basati su entropia, che hanno bypassato le nostre feature statiche.
        \end{itemize}
    \end{tcolorbox}
\end{frame}

% ---------------------------------------------------------
% 4.4 INFRASTRUCTURE & DEPLOYMENT
% ---------------------------------------------------------
\subsection*{Infrastructure \& Archiving}

\begin{frame}{Computational Landscape: La Sfida Hardware}
    \small
    Il training su dataset massivi (500k+ campioni) ha richiesto una gestione aggressiva delle risorse cloud.
    \vspace{0.1cm}

    \begin{columns}[c]
        \begin{column}{0.48\textwidth}
            \begin{tcolorbox}[enhanced, title=\faServer\ Infrastructure Setup, colframe=Obsidian, colback=white, fonttitle=\bfseries\footnotesize, drop shadow]
                \footnotesize
                \textbf{Platform:} Lightning AI Studio
                \par\vspace{0.1cm}
                \textbf{Hardware:} \textbf{NVIDIA L4} (24GB VRAM)
                \par\vspace{0.1cm}
                \textbf{Strategy:}
                \begin{itemize}
                    \setlength\itemsep{0em}
                    \item Training distribuito multi-account.
                    \item Checkpointing frequente anti-timeout.
                \end{itemize}
            \end{tcolorbox}
        \end{column}

        \begin{column}{0.48\textwidth}
            \begin{tcolorbox}[enhanced, title=\faStopwatch\ Training Cost, colframe=VividViolet, colback=white, fonttitle=\bfseries\footnotesize, drop shadow]
                \centering
                \LARGE \textbf{> 150h}
                \\ \footnotesize GPU Compute Time
                \vspace{0.1cm}
                \\ \scriptsize \faBolt\ \textit{High-Intensity Workload}
            \end{tcolorbox}
        \end{column}
    \end{columns}
    \pause
    \vspace{0.2cm}
    \begin{tcolorbox}[colback=gray!5, colframe=CyanGlow!80!black, title=\faMicrochip\ Optimization Strategy, fonttitle=\bfseries\footnotesize]
        \footnotesize
        Gestione del fine-tuning di UniXcoder su L4 tramite \textbf{Mixed Precision (FP16)} e \textbf{Gradient Accumulation} per simulare batch size elevati senza OOM (Out Of Memory).
    \end{tcolorbox}
\end{frame}

\begin{frame}{Model Archiving \& Reproducibility}
    \small
    Tutti gli artefatti sono stati versionati su Hugging Face (Private Hub) per audit e riproducibilità.
    \vspace{0.1cm}

    \begin{columns}[T]
        \begin{column}{0.38\textwidth}
            \begin{tcolorbox}[enhanced, title=\faDatabase\ Digital Assets, colframe=Obsidian, colback=white, fonttitle=\bfseries\footnotesize, drop shadow]
                \footnotesize
                \textbf{Account:}
                \\ \texttt{GiovanniIacuzzo02}
                \par\vspace{0.2cm}
                
                \textbf{Total Storage:}
                \\ $\approx$ \textbf{1.5 GB} (Weights + Logs)
                \par\vspace{0.2cm}
                
                \textbf{Artifacts:}
                \\ \faFileCode\ \texttt{model\_state.bin}
                \\ \faCogs\ \texttt{training\_meta.yaml}
                \\ \faKey\ \texttt{tokenizer.json}
                \par\vspace{0.2cm}
                
                \centering
                \fcolorbox{gray}{gray!10}{\scriptsize \faLock\ Private Repo Access}
            \end{tcolorbox}
        \end{column}

        \begin{column}{0.60\textwidth}
            \begin{tcolorbox}[enhanced, title=\faSitemap\ Repository Structure, colframe=VividViolet, colback=white, fonttitle=\bfseries\footnotesize, drop shadow]
                \scriptsize
                \setlength{\leftmargini}{1em}
                
                \textbf{1. Subtask A (Single Model)}
                \\ \texttt{SemEval2026\_Task13\_subtask\_A}
                \begin{itemize}
                    \item[\faFile] \texttt{best\_model/model\_state.bin}
                \end{itemize}
                \vspace{0.1cm}

                \textbf{2. Subtask B (Cascade Architecture)}
                \\ \texttt{SemEval2026\_Task13\_subtask\_B}
                \begin{itemize}
                    \setlength\itemsep{0em}
                    \item[\faFolderOpen] \textbf{\texttt{binary/}} $\to$ \textit{Gatekeeper Weights}
                    \item[\faFolderOpen] \textbf{\texttt{families/}} $\to$ \textit{Multiclass Weights}
                \end{itemize}
                \vspace{0.1cm}

                \textbf{3. Subtask C (Modular System)}
                \\ \texttt{SemEval2026\_Task13\_subtask\_C}
                \begin{itemize}
                    \setlength\itemsep{0em}
                    \item[\faFolderOpen] \textbf{\texttt{stage1\_binary\_human\_machine/}}
                    \item[\faFolderOpen] \textbf{\texttt{stage2\_machine\_attribution/}}
                \end{itemize}
            \end{tcolorbox}
        \end{column}
    \end{columns}
\end{frame}