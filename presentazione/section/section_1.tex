% --- SEZIONE 1: CONTESTO E OBIETTIVI ---
\section{Introduzione e Obiettivi}

% ---------------------------------------------------
% SLIDE 1: IL CONTESTO
% ---------------------------------------------------
\begin{frame}{Il Nuovo Paradigma: GenAI Code}
    \vspace{0.2cm}
    
    % Contenitore superiore
    \begin{tcolorbox}[
        colback=white, colframe=Obsidian, 
        title=\textbf{\faExclamationTriangle \ \ The Rise of LLMs},
        fonttitle=\bfseries\normalsize, coltitle=white,
        enhanced, attach boxed title to top left={xshift=0.5cm, yshift=-2mm},
        boxed title style={colback=VividViolet, sharp corners},
        boxrule=0.5pt, drop fuzzy shadow
    ]
        L'evoluzione di modelli come \textbf{GPT-4, Llama-3 e StarCoder} ha reso il codice sintetico indistinguibile da quello umano. Questo fenomeno introduce rischi critici per la \textbf{sicurezza software}, il \textbf{copyright} e l'integrità dei sistemi.
    \end{tcolorbox}
    
    \vspace{0.3cm}

    % Visualizzazione Concettuale
    \begin{center}
    \begin{tikzpicture}
        % 1. Nodo Umano
        \node[align=center] (human) at (0,0) {
            \color{Obsidian}\Huge\faUser\\
            \small\textbf{Human Code}\\
            \scriptsize(Creativity, Bugs)}
        ;
        
        % 2. Nodo AI
        \node[align=center] (ai) at (8,0) {
            \color{NeonPurple}\Huge\faRobot\\
            \small\textbf{GenAI Code}\\
            \scriptsize(Speed, Patterns)}
        ;
        
        % 3. Freccia "Indistinguishable"
        \onslide<2->{
            \draw[->, line width=2pt, draw=gray!30] (human) -- (ai) node[midway, above] {\color{DarkText}\textbf{Indistinguishable?}};
        }
        
        % 4. Box Soluzione
        \onslide<3->{
            \node[draw=NeonPurple, fill=white, rounded corners, inner sep=10pt, drop shadow] at (4, -1.5) {
                \begin{minipage}{0.6\textwidth}
                    \centering
                    \color{VividViolet}\textbf{SemEval-2026 Task 13}\\
                    \color{gray}\scriptsize Detect $\cdot$ Attribute $\cdot$ Analyze
                \end{minipage}
            };
        }
    \end{tikzpicture}
    \end{center}
\end{frame}

% ---------------------------------------------------
% SLIDE 2: LE 3 SFIDE CHIAVE
% ---------------------------------------------------
\begin{frame}{Obiettivi della Competizione}
    \vspace{0.1cm}
    La Task 13 risponde a tre esigenze fondamentali dell'industria moderna:
    \vspace{0.3cm}
    
    \begin{columns}[t, onlytextwidth]
        
        % CARD 1: SECURITY
        \column{0.32\textwidth}
        \onslide<1->{
            \begin{tcolorbox}[
                enhanced, colback=white, colframe=gray!10, 
                coltitle=Obsidian, title=\centering\Large\faLock, 
                fonttitle=\bfseries, arc=5pt,
                borderline={0pt}{0pt}{white},
                shadow={2mm}{-2mm}{0mm}{black!5},
                top=5pt
            ]
                \centering
                \textbf{\color{VividViolet}Sicurezza}\\
                \vspace{0.2cm}
                \scriptsize
                Prevenire vulnerabilità introdotte da codice generato (Hallucinations) in sistemi critici.
            \end{tcolorbox}
        }

        % CARD 2: ATTRIBUTION
        \column{0.32\textwidth}
        \onslide<2->{
            \begin{tcolorbox}[
                enhanced, colback=white, colframe=gray!10, 
                coltitle=Obsidian, title=\centering\Large\faCopyright,
                fonttitle=\bfseries, arc=5pt,
                borderline={0pt}{0pt}{white},
                shadow={2mm}{-2mm}{0mm}{black!5},
                top=5pt
            ]
                \centering
                \textbf{\color{VividViolet}Attribuzione}\\
                \vspace{0.2cm}
                \scriptsize
                Identificare il modello specifico (es. Llama vs OpenAI) per licensing e proprietà intellettuale.
            \end{tcolorbox}
        }

        % CARD 3: HYBRID ANALYSIS
        \column{0.32\textwidth}
        \onslide<3->{
            \begin{tcolorbox}[
                enhanced, colback=white, colframe=gray!10, 
                coltitle=Obsidian, title=\centering\Large\faCodeBranch,
                fonttitle=\bfseries, arc=5pt,
                borderline={0pt}{0pt}{white},
                shadow={2mm}{-2mm}{0mm}{black!5},
                top=5pt
            ]
                \centering
                \textbf{\color{VividViolet}Analisi Ibrida}\\
                \vspace{0.2cm}
                \scriptsize
                Gestire scenari complessi: codice misto Uomo-AI, Refactoring e Codice Avversario.
            \end{tcolorbox}
        }

    \end{columns}
\end{frame}

% ---------------------------------------------------
% SLIDE 3: STRUTTURA DEI SUBTASKS
% ---------------------------------------------------
\begin{frame}{Struttura del Progetto: I 3 Moduli}
    
    \vspace{0.1cm}
    Una pipeline unificata per affrontare i tre livelli di complessità del task:
    \vspace{0.2cm}

    % SUBTASK A
    \onslide<1->{
        \begin{tcolorbox}[
            enhanced, colback=white, 
            frame style={left color=VividViolet, right color=NeonPurple},
            coltitle=white, title=\textbf{Subtask A $\cdot$ Binary Detection},
            fonttitle=\bfseries\scriptsize, 
            leftrule=4mm, boxrule=0pt, arc=2pt,
            drop fuzzy shadow,
            top=2pt, bottom=2pt
        ]
            \begin{columns}
                \column{0.08\textwidth} \centering \color{white}\faAdjust
                \column{0.92\textwidth} 
                \scriptsize 
                \textbf{Human vs Machine.} \\
                Classificazione binaria su 500K sample. Include scenari \textit{Unseen Languages} (es. Go, PHP) e \textit{Unseen Domains}.
            \end{columns}
        \end{tcolorbox}
    }
    \vspace{0.05cm}

    % SUBTASK B
    \onslide<2->{
        \begin{tcolorbox}[
            enhanced, colback=white, 
            frame style={left color=Obsidian, right color=Charcoal},
            coltitle=white, title=\textbf{Subtask B $\cdot$ Attribution},
            fonttitle=\bfseries\scriptsize, 
            leftrule=4mm, boxrule=0pt, arc=2pt,
            drop fuzzy shadow,
            top=2pt, bottom=2pt
        ]
            \begin{columns}
                \column{0.08\textwidth} \centering \color{white}\faFingerprint
                \column{0.92\textwidth} 
                \scriptsize 
                \textbf{Multi-Class (11 Families).} \\
                Identificazione fine-grained dell'autore (Human, GPT, Llama, Mistral, ecc.). Sfida: Forti sbilanciamenti di classe.
            \end{columns}
        \end{tcolorbox}
    }
    \vspace{0.05cm}

    % SUBTASK C
    \onslide<3->{
        \begin{tcolorbox}[
            enhanced, colback=white, 
            frame style={left color=CyanGlow!80!blue, right color=CyanGlow},
            coltitle=white, title=\textbf{Subtask C $\cdot$ Hybrid Analysis},
            fonttitle=\bfseries\scriptsize, 
            leftrule=4mm, boxrule=0pt, arc=2pt,
            drop fuzzy shadow,
            top=2pt, bottom=2pt
        ]
            \begin{columns}
                \column{0.08\textwidth} \centering \color{white}\faRandom
                \column{0.92\textwidth} 
                \scriptsize 
                \textbf{Mixed \& Adversarial (4 Classi).} \\
                Classificazione complessa: Human, Machine, Hybrid (Mixed) e Adversarial (Codice offuscato). 900K sample.
            \end{columns}
        \end{tcolorbox}
    }

\end{frame}