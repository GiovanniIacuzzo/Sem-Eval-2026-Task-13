% =========================================================
% SEZIONE 5: CONCLUSIONI
% =========================================================
\section{Conclusioni}

\begin{frame}{Bilancio Finale: "The Good, The Bad and The Overfit"}
    Un percorso di 150+ ore di GPU che ha insegnato una dura lezione: \textbf{Il Validation Set mente.}
    \vspace{0.1cm}

    \begin{columns}[t]
        \begin{column}{0.48\textwidth}
            \begin{tcolorbox}[enhanced, title=\faThumbsDown\ The "Reality Check" (A \& C), colframe=red!60!black, colback=red!5, fonttitle=\bfseries\footnotesize, drop shadow]
                \footnotesize
                \textbf{L'Illusione:} Vedere F1 $> 0.99$ in locale.
                \par\vspace{0.1cm}
                \textbf{La Realtà:} Il modello aveva memorizzato la sintassi (Python/Java) invece della semantica. Di fronte a Go/PHP o ad attacchi massivi, \textbf{è crollato}.
                \par\vspace{0.1cm}
                \textit{"Frustrante? Sì. Istruttivo? Assolutamente."}
            \end{tcolorbox}
        \end{column}

        \begin{column}{0.48\textwidth}
            \begin{tcolorbox}[enhanced, title=\faThumbsUp\ The Success Story (Task B), colframe=CyanGlow!80!blue, colback=CyanGlow!5, fonttitle=\bfseries\footnotesize, drop shadow]
                \footnotesize
                \textbf{Il Riscatto:} L'approccio a Cascata + SupCon ha funzionato.
                \par\vspace{0.1cm}
                \textbf{Risultato:} 7\textsuperscript{o} posto su Kaggle, incollato alla Top 5.
                \par\vspace{0.1cm}
                \textit{Dimostra che su domini circoscritti (11 famiglie), l'architettura è solida.}
            \end{tcolorbox}
        \end{column}
    \end{columns}

    \vspace{0.2cm}
    \centering
    \begin{tcolorbox}[colback=white, colframe=gray!50, width=0.9\textwidth, boxrule=0.5pt, arc=2mm]
        \centering \footnotesize \faQuoteLeft\ \textit{Non è il risultato che speravo, ma è l'esperienza tecnica che mi serviva.} \faQuoteRight
    \end{tcolorbox}
\end{frame}

\begin{frame}{Legacy: Costruiamo il "Kore AI Lab"}
    Questa competizione è stata brutale ma formativa. Non voglio che l'esperienza vada persa.
    \vspace{0.1cm}

    \begin{columns}[c]
        \begin{column}{0.30\textwidth}
            \centering
            \begin{tikzpicture}
                \node[circle, draw=VividViolet, line width=2pt, inner sep=2pt, fill=white, drop shadow] (icon) {
                    \includegraphics[width=2.5cm, height=2.5cm, keepaspectratio]{img/Logo_UKE.png}
                };
                \node[below=0.2cm of icon, font=\bfseries\small\color{Obsidian}] {Kore University};
            \end{tikzpicture}
        \end{column}

        \begin{column}{0.65\textwidth}
            \begin{tcolorbox}[enhanced, title=\faHandshake\ Proposta per il Futuro, colframe=VividViolet, colback=white, fonttitle=\bfseries\footnotesize, drop shadow]
                \footnotesize
                Propongo la creazione di un \textbf{gruppo studentesco competitivo} per le prossime challenge (SemEval, Kaggle).
                \vspace{0.1cm}
                
                \textbf{Il mio contributo per i prossimi studenti:}
                \begin{itemize}
                    \setlength\itemsep{0em}
                    \item \textbf{Mentorship:} Setup ambienti Cloud e gestione GPU L4.
                    \item \textbf{Assets:} Pipeline di training pronte e debuggate.
                    \item \textbf{Strategy:} Evitare i miei errori (Overfitting).
                \end{itemize}
            \end{tcolorbox}
        \end{column}
    \end{columns}

    \centering
    \begin{tcolorbox}[colback=VividViolet!10, colframe=VividViolet, boxrule=0pt, arc=2mm]
        \centering \bfseries \color{Obsidian} \small
        "L'obiettivo non è che io vinca da solo oggi, ma che l'anno prossimo qualcuno della Kore faccia meglio di me."
    \end{tcolorbox}
\end{frame}

\begin{frame}[plain]
    \centering
    \vspace{0.5cm}
    
    {\fontsize{32}{40}\selectfont \color{Obsidian}\textbf{Grazie per l'Attenzione}}
    
    \vspace{0.6cm}
    
    \begin{tcolorbox}[enhanced, colback=gray!5, colframe=Obsidian, width=0.9\textwidth, arc=3mm, drop shadow]
        \centering
        \vspace{0.2cm}
        
        \begin{columns}[t]
            \begin{column}{0.48\textwidth}
                \centering
                \large \faGithub\ \textbf{GitHub} \\
                \texttt{GiovanniIacuzzo}
                
                \vspace{0.4cm}
                
                \large \faKaggle\ \textbf{Kaggle} \\
                \texttt{giovanniiacuzzo}
            \end{column}
            
            \begin{column}{0.48\textwidth}
                \centering
                \large \faBrain\ \textbf{Hugging Face} \\
                \texttt{GiovanniIacuzzo02}
                
                \vspace{0.4cm}
                
                \large \faEnvelope\ \textbf{Email} \\
                \small \texttt{giovanni.iacuzzo@unikore.it}
            \end{column}
        \end{columns}
        
        \vspace{0.2cm}
    \end{tcolorbox}

    \vspace{0.5cm}
    \small \textit{Code \& Models available via Private Access (Audit Ready)}

    \vfill
    \footnotesize \textit{SemEval 2026 Task 13 - Project Presentation}
    \vspace{0.5cm}
\end{frame}