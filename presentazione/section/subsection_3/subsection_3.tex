\subsection*{Subtask C: Hybrid \& Adversarial Scenarios}

% SLIDE C.1: DEFINIZIONE
\begin{frame}{Subtask C: La Sfida delle "Zone Grigie"}
    Oltre alla distinzione binaria, il modello deve classificare scenari ad alta ambiguità (4 Classi).
    \vspace{0.3cm}
    \pause
    \begin{columns}[t]
        \begin{column}{0.48\textwidth}
            \begin{tcolorbox}[enhanced, colframe=gray!50, colback=white, title=\faCheckCircle\ Classi "Pure", fonttitle=\bfseries\footnotesize, height=3.2cm, drop shadow]
                \footnotesize
                \begin{itemize}
                    \item \textbf{Human (0):} Logica imperfetta.
                    \item \textbf{AI-Generated (1):} Pattern statistici puliti.
                \end{itemize}
                \vspace{0.1cm}
                \centering \textit{Facilmente separabili.}
            \end{tcolorbox}
        \end{column}%
        \pause
        \begin{column}{0.48\textwidth}
            \begin{tcolorbox}[enhanced, colframe=Obsidian, colback=white, title=\faExclamationTriangle\ Classi "Rumorose", fonttitle=\bfseries\footnotesize, height=3.2cm, drop shadow]
                \footnotesize
                \begin{itemize}
                    \item \textbf{\faUserCog\ Hybrid (2):} AI + Umano. \textit{Discontinuità stilistica.}
                    \item \textbf{\faMask\ Adversarial (3):} Offuscato. \textit{Alta entropia.}
                \end{itemize}
            \end{tcolorbox}
        \end{column}%
    \end{columns}
    \pause
    \vspace{0.3cm}
    \centering
    \begin{tcolorbox}[colback=VividViolet!5, colframe=VividViolet, width=0.95\textwidth, arc=2mm, boxrule=0.5pt]
        \centering \small 
        \faCrosshairs\ \textbf{Obiettivo:} Robustezza. Riconoscere la manipolazione.
    \end{tcolorbox}
\end{frame}

% SLIDE C.2: TRAINING STRATEGY
\begin{frame}{Subtask C: Advanced Training Strategy}
    Per gestire classi ibride e sbilanciate, utilizziamo una combinazione pesata di loss functions.
    \pause
    \vspace{0.3cm}
    \begin{tcolorbox}[enhanced, title={Hybrid Loss Function ($\alpha=0.5$)}, colframe=Obsidian, colback=white, fonttitle=\bfseries\small, drop shadow]
        $$ \mathcal{L}_{Total} = \mathcal{L}_{Focal} + \alpha \cdot \mathcal{L}_{SupCon} $$
    \end{tcolorbox}

    \vspace{0.2cm}
    \begin{columns}[T]
        \begin{column}{0.48\textwidth}
            \textbf{1. Focal Loss ($\gamma=2.0$)}
            \vspace{0.1cm}
            \begin{itemize}
                \footnotesize
                \item Penalizza gli esempi "Hard" (Hybrid).
                \item \textit{Smoothing:} $\epsilon=0.1$ per evitare overconfidence.
            \end{itemize}
        \end{column}%
        \begin{column}{0.48\textwidth}
            \textbf{2. SupCon Loss ($\tau=0.07$)}
            \vspace{0.1cm}
            \begin{itemize}
                \footnotesize
                \item \textbf{Push:} Massimizza distanza Human-Hybrid.
                \item \textbf{Pull:} Compatta i cluster della stessa classe.
            \end{itemize}
        \end{column}%
    \end{columns}
\end{frame}

% SLIDE C.3: FEATURE ADVERSARIAL
\begin{frame}{Subtask C: Rilevamento Offuscamento}
    Estrattore parallelo (\texttt{ProcessPoolExecutor}) per feature di sicurezza.
    \vspace{0.2cm}
    \pause
    \begin{columns}[T]
        \begin{column}{0.48\textwidth}
            \begin{tcolorbox}[enhanced, colback=white, colframe=VividViolet, title=\faRandom\ Entropy \& Complexity, fonttitle=\bfseries\small, drop shadow]
                \begin{itemize}
                    \setlength\itemsep{0.6em}
                    \item \textbf{Shannon Entropy:} 
                    \\ \scriptsize Picchi di entropia indicano rinomina casuale delle variabili.
                    \item \textbf{Nesting Depth:} 
                    \\ \scriptsize L'offuscamento crea nesting innaturali di \texttt{\{\}}.
                \end{itemize}
            \end{tcolorbox}
        \end{column}%
        \pause
        \begin{column}{0.48\textwidth}
            \begin{tcolorbox}[enhanced, colback=white, colframe=Obsidian, title=\faShield*\ Payload Detection, fonttitle=\bfseries\small, drop shadow]
                \begin{itemize}
                    \setlength\itemsep{0.6em}
                    \item \textbf{Long String Ratio:} 
                    \\ \scriptsize Rilevamento Regex \texttt{"[\textasciicircum"]\{50,\}"} di stringhe sospette (es. Base64).
                    \item \textbf{Keyword Density:} 
                    \\ \scriptsize Diluizione delle keyword sintattiche standard.
                \end{itemize}
            \end{tcolorbox}
        \end{column}%
    \end{columns}
    
    \vspace{0.3cm}
    \centering \footnotesize
    \faCogs\ \textit{Implementation Note:} Feature normalizzate ed estratte con multiprocessing.
\end{frame}