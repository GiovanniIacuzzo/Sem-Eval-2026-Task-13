{
    \setbeamercolor{background canvas}{bg=PaperWhite}
    % Forziamo Beamer a considerare il testo nero in questo contesto
    \setbeamercolor{section in toc}{fg=Obsidian} 
    \setbeamercolor{section number projected}{bg=VividViolet,fg=white}
    
    % --- FIX FONDAMENTALE PER I COLORI ---
    % Reimpostiamo il colore dei link SOLO per questa slide per evitare che 
    % tornino bianchi ereditando le impostazioni globali
    \hypersetup{linkcolor=Obsidian}

    % --- DESIGN DEL "BOTTONE" DELL'INDICE ---
    \setbeamertemplate{section in toc}{
        \begin{tikzpicture}
            \node[
                draw=gray!20,
                fill=white,
                rounded corners=3pt,
                inner sep=8pt,
                text width=0.36\paperwidth,
                align=left,
                drop shadow={opacity=0.05, shadow xshift=2pt, shadow yshift=-2pt},
                % --- MODIFICA QUI: Forziamo il colore del testo del nodo ---
                text=Obsidian 
            ] (box) {
                % Numero Sezione
                {\color{VividViolet}\bfseries\large \inserttocsectionnumber}
                \hspace{0.2cm}
                % Linea verticale
                {\color{gray!30}\vrule width 1pt} 
                \hspace{0.2cm}
                % Titolo Sezione
                % Usiamo \textcolor esplicito per sovrascrivere qualsiasi impostazione beamer
                {\color{Obsidian}\bfseries \inserttocsection}
            };
        \end{tikzpicture}
        \par\vspace{0.3cm}
    }

    \begin{frame}[plain]
        
        % --- INTESTAZIONE ---
        \begin{tikzpicture}[remember picture, overlay]
            \fill[Obsidian] (current page.north west) rectangle ([yshift=-2.5cm]current page.north east);
            
            \node[anchor=west] at ([xshift=1cm, yshift=-1.2cm]current page.north west) {
                \fontsize{32}{36}\selectfont \bfseries \color{white} Roadmap
            };
            
            \node[anchor=east] at ([xshift=-1cm, yshift=-1.2cm]current page.north east) {
                \color{CyanGlow} \faMap \ SemEval-2026
            };
            
            \fill[VividViolet] ([yshift=-2.5cm]current page.north west) rectangle ([yshift=-2.6cm]current page.north east);
        \end{tikzpicture}
    
        \vspace{2.5cm} 
        
        % --- INDICE A DUE COLONNE ---
        \begin{columns}[t]
            \begin{column}{0.48\textwidth}
                % COLONNA 1: Sezioni da 1 a 3
                \tableofcontents[sections={1-3}] 
            \end{column}
            
            \begin{column}{0.48\textwidth}
                % COLONNA 2: Sezioni da 4 alla fine
                \tableofcontents[sections={4-10}]
            \end{column}
        \end{columns}
        
    \end{frame}
}